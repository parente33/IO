\documentclass[a4paper,12pt]{article}

\usepackage[utf8]{inputenc}       % Codificação
\usepackage[T1]{fontenc}          % Saida de caracteres em portugues
\usepackage[portuguese]{babel}    % Idioma
\usepackage{amsmath,amssymb,amsfonts} % Pacotes matemáticos
\usepackage{lscape}               % se precisar de páginas em paisagem
\usepackage{graphicx}             % se precisar de inserir figuras
\usepackage{geometry}             % para ajustar margens
\usepackage{booktabs}             % para tabelas mais bonitas
\usepackage{array}                % para colunas customizadas em tabelas
\usepackage{tikz}
\usepackage{listings}
\usepackage{xcolor, colortbl}
\usepackage{textcomp}
\usepackage{float}
\usetikzlibrary{arrows}

\lstset{
    basicstyle=\ttfamily\footnotesize,   % Fonte monoespaçada pequena
    breaklines=true,                      % Quebra de linha automática
    frame=single,                          % Adiciona uma caixa em volta
    backgroundcolor=\color{gray!10},       % Fundo levemente cinza
    keywordstyle=\color{blue},             % Palavras-chave em azul
    commentstyle=\color{green!50!black},   % Comentários em verde escuro
    rulecolor=\color{black},               % Cor da moldura
}
\geometry{a4paper,margin=2.5cm}

\title{
    Investigação Operacional\protect\\
}
\author{André Santos - a106854
    \and Daniel Parente - a107363
    \and Pedro Ferreira - a107292
}

\date{Ano Letivo 2024/2025}

\setlength{\parindent}{0pt}
\begin{document}
\vspace*{\fill} % Espaço antes para centralizar verticalmente

\begin{minipage}{0.9\linewidth}
    \centering
    \includegraphics[width=0.4\textwidth]{images.jpg}\par
    \vspace{1cm}

    {\scshape\LARGE Universidade do Minho} \par
    \vspace{0.6cm}

    {\scshape\Large Licenciatura em Engenharia Informática} \par
    \maketitle
    \vspace{1cm}
\end{minipage}

\vspace*{\fill}
\pagebreak

\section*{0. Valores a trabalhar}
De acordo com os dados do enunciado, e tal como já tinha sido o caso no primeiro projeto, o valor de $xABCDE$ a trabalhar será $107363$, o maior número de inscrição dos membros do grupo. Podemos obviamente concluir que: $x = 1$, $A = 0$, $B = 7$, $C = 3$, $D = 6$ e $E = 3$.

\vspace{0.5em}

Vamos calcular o valor de $k$, sabendo que equivale ao resto da divisão de $DE$ por $7$. Ou seja, o número natural tal que $63 = x * 7 + k$. Daqui tiramos que $63 = 9 * 7 + 0$. Como $k = 0$, os vértices origem $O$ e destino $D$ são, pela tabela no enunciado, $1$ e $4$, respetivamente.

\vspace{0.5em}

As capacidades dos outros vértices foram também calculadas. Essa informação encontra-se calculada a seguir:

\subsection*{Capacidade dos vértices}
\begin{itemize}
    \item Vértice $1$: $+\infty$ \textit{(exceção mencionada no enunciado)};
    \item Vértice $2$: $10 \times (B + D + 1) = 10 \times (7 + 6 + 1) = 10 \times 14 = 140$;
    \item Vértice $3$: $10 \times (C + 1) = 10 \times (3 + 1) = 10 \times 4 = 40$;
    \item Vértice $4$: $+\infty$ \textit{(exceção mencionada no enunciado)};
    \item Vértice $5$: $10 \times (E + 1) = 10 \times (3 + 1) = 10 \times 4 = 40$;
    \item Vértice $6$: $10 \times (D + E + 1) = 10 \times (6 + 3 + 1) = 10 \times 10 = 100$;
\end{itemize}

A qual se pode sumarizar na Tabela 1:~\ref{tab:capacidades}

\begin{table}[h!]
\centering
\label{tab:capacidades}
\begin{tabular}{@{}c c@{}}
\toprule
\textbf{Vértice} & \textbf{Capacidade} \\
\midrule
$1$ & $+\infty$ \\
$2$ & $140$       \\
$3$ & $40$       \\
$4$ & $+\infty$       \\
$5$ & $40$       \\
$6$ & $100$       \\
\bottomrule
\end{tabular}
\caption{Capacidade dos vértices.}
\end{table}

Com essa informação disponível, o grafo a usar para o problema está representado a seguir, onde $O$ e $D$ são os vértices \textit{origem} e \textit{destino}, respetivamente, e o número por cima de cada vértice representa a sua capacidade:

\begin{figure}[H]
    \centering
    \begin{tikzpicture}[scale=1, every node/.style={draw, circle, minimum size=7mm, inner sep=0pt}]
        \node (O) at (0,1) {O};
        \node (2) at (2,2) {2};
        \node (3) at (2,0) {3};
        \node (D) at (4,2) {D};
        \node (5) at (4,0) {5};
        \node (6) at (6,1) {6};

        \node[above=5pt, draw=none] at (O) {\small $+\infty$};
        \node[above=5pt, draw=none] at (2) {\small $140$};
        \node[above=14pt, right=-1pt, draw=none] at (3) {\small $40$};
        \node[above=5pt, draw=none] at (D) {\small $+\infty$};
        \node[above=14pt, right=-1pt, draw=none] at (5) {\small $40$};
        \node[above=5pt, draw=none] at (6) {\small $100$};

        \draw (O) -- (2);
        \draw (O) -- (3);
        \draw (2) -- (D);
        \draw (2) -- (3);
        \draw (3) -- (5);
        \draw (D) -- (5);
        \draw (D) -- (6);
        \draw (5) -- (6);
    \end{tikzpicture}
    \caption{Grafo representativo da rede de fluxos}
    \label{fig:network-graph}
\end{figure}

\section*{1. Formulação do Problema}
\subsection*{1.1 Descrição do problema}
O problema apresentado neste trabalho prático traduz-se num problema de otimização de fluxos em redes, onde se pretende determinar o fluxo máximo que pode ser transportado de um vértice de origem (aqui denominado $O$), para um vértice de destino ($D$), respeitando as restrições impostas pelas capacidades dos vértices intermediários.

\vspace{0.5em}

Ao contrário do tradicional problema de fluxo máximo em que as restrições são normalmente colocadas sobre as arestas, aqui o foco recai sobre as quantidades internas dos vértices, que representam limites na quantidade de fluxo que pode passar por cada um.

\subsection*{1.2 Objetivo}
O objetivo central deste problema é determinar o valor máximo de fluxo que pode ser enviado do vértice origem $O$ para o vértice destino $D$, sem exceder as capacidades de processamento dos vértices intermediários (vertíces 2, 3, 5 e 6). Para isso, é necessário encontrar a melhor forma de distribuir o fluxo ao longo da rede de forma eficiente, utilizando os vários caminhos disponíveis, enquanto se respeitam todas as restrições impostas.

\subsection*{1.3 Composição da rede}
A rede é, assim, composta por:
\begin{itemize}
    \item Vértices: Representam pontos de decisão ou nós de um sistema de fluxo.
    \begin{itemize}
        \item O vértice O é o ponto de origem do fluxo
        \item O vértice D é o ponto de destino do fluxo
        \item Os vértices 2,3,5 e 6 são vértices intermediários com capacidades limitadas.
    \end{itemize}
    \item Arestas: Representam as conexões entre os vértices. Cada aresta tem uma capacidade virtualmente infinita. São também bidirecionais.
\end{itemize}

Capacidade dos vértices:
\begin{itemize}
    \item Os vértices 2, 3, 5 e 6 têm capacidades definidas:
        \begin{itemize}
            \item Vértice 2: Capacidade de 140 unidades de fluxo;
            \item Vértice 3: Capacidade de 40 unidades de fluxo;
            \item Vértice 5: Capacidade de 40 unidades de fluxo;
            \item Vértice 6: Capacidade de 100 unidades de fluxo.
        \end{itemize}
    \item Os vértices 1 e 4 (O e D, respetivamente) têm capacidades infinitas.
\end{itemize}

\subsection*{1.4 Relação com um sistema real}
Este tipo de problema é particularmente relevante em contextos reais onde o congestionamento ou a capacidade de processamento num ponto específico da rede (e não do caminho) impõe o verdadeiro limite ao fluxo total - por exemplo, em redes logísticas, redes de dados, ou sistemas de transportes urbanos.

\subsection*{1.5 Explicação dos valores das capacidades}
Os valores das capacidades dos vértices foram definidos com base no maior número de inscrição nos membros do grupo - neste caso, $xABCDE = 107363$. A correspondência entre os dígitos do número de inscrição e os parâmetros das fórmulas é a seguinte:
\begin{itemize}
    \item $x = 1$
    \item $A = 0$
    \item $B = 7$
    \item $C = 3$
    \item $D = 6$
    \item $E = 3$
\end{itemize}
As capacidades dos vértices intermediários foram calculados com base nas informações especificadas no enunciado. Estas foram:
\begin{itemize}
    \item Vértice 2: $10 \times (B + D + 1) = 10 \times (7  6 + 1) = 140$
    \item Vértice 3: $10 \times (C + 1) = 10 \times (3 + 1) = 40$
    \item Vértice 5: $10 \times (E + 1) = 10 \times (3 + 1) = 40$
    \item Vértice 6: $10 \times (D + E + 1) = 10 \times (6 + 3 + 1) = 100$
\end{itemize}
Já os vértices $0$ (1) e $D$ (4) foram definidos como tendo capacidade infinita, pois representam os extremos da rede.

\subsection*{1.6 Explicação dos valores dos consumos}
Apesar de o enunciado não referir diretamente consumos em termos clássicos (como perdas ou custos por vérice), podemos interpretar este tópico em dois sentidos complementares:
\begin{enumerate}
    \item Consumo da capacidade dos vértices: Cada vértice com capacidade limitada "consome" parte do fluxo à medida que este o atravessa. Assim, o consumo corresponde ao fluxo que efetivamente passa por esse vértice, o qual nunca pode ultrapassar o limite máximo atribuído. Por exemplo:
    \begin{itemize}
        \item Se o vértice 2 tem capacidade 140, e o fluxo que o atravessa for 120, então o consumo é de 120, e ainda há margem para mais 20 unidades de fluxo.
        \item Caso o fluxo seja superior a 140, então esse vértice torna-se um "gargalo", limitando a passagem.
    \end{itemize}
    \item Interpretação do fluxo como consumo de recursos: Em problemas reais (logística, comunicações, etc.), o fluxo pode representar bens, energia ou dados. O seu trajeto pela rede representa o consumo de recursos - por exemplo, tempo de processamento, largura de banda, ou energia. Assim, as capacidades funcionam como restrições ao consumo máximo que cada vértice pode suportar.
\end{enumerate}
Portanto, neste modelo, o termo "consumo" pode ser entendido como a utilização da capacidade disponível em cada vértice. O modelo assume que não há perdas de fluxo (rede conservativa), ou seja, o que entra num vértice (menos o que sai) corresponde exatamente ao fluxo que ali passa, excetuando-se origem e destino.

\subsection*{1.7 Apresentação da coerência global do modelo}
O modelo proposto apresenta uma estrutura coerente e logicamente bem fundamentada, com os seguintes pontos fortes:
\begin{itemize}
    \item Personalização e variabilidade: O uso de um número de inscrição como base para determinar parâmetros garante que cada grupo trabalhe com um caso único, reduzindo a probabilidade de duplicações.
    \item Consistência matemática: As fórmulas utilizadas para calcular as capacidades são simples, mas eficazes, garantindo valores realistas que não são nem demasiado elevados (o que tornaria o problema trivial), nem demasiado baixos (o que poderia torná-lo insolúvel ou desproporcional).
    \item Topologia da rede bem definida: A estrutura do grafo permite múltiplos caminhos do vértice de origem para o de destino, o que possibilita o estudo de como o fluxo se redistribui na presença de restrições. A inclusão de ciclos ou ligações cruzadas (como de $3 \rightarrow 5$ e $D \rightarrow 5$) torna o problema mais interessante do ponto de vista algorítmico.
\end{itemize}
Em resumo, o modelo é coerente, aplicável, suficientemente desafiante e adequado para ilustrar técnicas avançadas de otimização de redes.

\section*{2. Formulação do modelo}
Neste trabalho, será efetuada a construção de dois modelos distintos para o problema de fluxo máximo apresentado: um utilizando o Relax4, e outro recorrendo ao LP Solver. A escolha de desenvolver ambos os modelos prende-se com razões de validação e complementaridade: o modelo implementado no Relax4 é o principal, dado que este solver é otimizado especificamente para problemas de fluxo em redes e oferece elevada eficiência computacional.

\vspace{0.5em}

No entanto, o LP Solver será utilizado como ferramenta auxiliar para validar a correção estrutural e lógica do modelo, já que permite uma formulação explícita do problema em termos de programação linear, facilitando a leitura e análise detalhada das restrições e variáveis. Esta abordagem dual garante maior robustez à solução final, promovendo simultaneamente a compreensão teórica e prática da modelação do problema.

\subsection*{2.1 Modelo no solver Relax4}
Para utilizar o Relax-4 num problema de fluxo máximo com capacidade nos vértices, aplica-se a seguinte conversão clássica:

\begin{itemize}
    \item Divisão de vértices: Cada vértice original $v$ é dividido em dois nós, $v_{in}$ e $v_{out}$, ligados por um arco com capacidade:
    \begin{equation*}
        u_{v_{in}, v_{out}} = \text{capacidade}(v)
    \end{equation*}

    \item Arestas originais: Se existia uma aresta original $u \rightarrow v$, cria-se o arco $u_{out} \rightarrow v_{in}$ com capacidade infinita (por exemplo, $10^6$) e custo zero.

    \item Arco com custo -1: Acrescenta-se o arco:
    \begin{equation*}
         D_{out} \rightarrow O_{in} \quad \text{com custo} = -1, \quad \text{capacidade infinita}
    \end{equation*}

    \item Ofertas/procuras: Todos os nós têm balanço 0 (fluxo circula em ciclos fechados).
\end{itemize}

O valor ótimo devolvido será algo como:

\begin{equation*}
    \text{(arco -1)} \quad F^*, \quad \text{OPTIMAL COST} = -F^*
\end{equation*}

\vspace{0.5em}

E, assim, o grafo transformado seria assim:
\begin{figure}[H]
    \centering
    \begin{tikzpicture}[scale=1, every node/.style={draw, circle, minimum size=7mm, inner sep=0pt}]

    % Vértices
    \node (0) at (0,3.5) {1}; % O
    \node (1) at (0,2.5) {2}; % O
    \node (2) at (2,4.5) {3}; % 2
    \node (3) at (3,4.5) {4}; % 2
    \node (4) at (2,1.5) {5}; % 3
    \node (5) at (3,1.5) {6}; % 3
    \node (6) at (5,4.5) {7}; % D
    \node (7) at (6,4.5) {8}; % D
    \node (8) at (5,1.5) {9}; % 5
    \node (9) at (6,1.5) {10}; % 5
    \node (10) at (8,3.5) {11}; % 6
    \node (11) at (8,2.5) {12}; % 6

    % Capacidades    
    %\node[above=5pt, draw=none] at (0) {\small $+\infty$};
    %\node[below=5pt, draw=none] at (1) {\small $+\infty$};
    %\node[above=5pt, draw=none] at (2) {\small $140$};
    %\node[above=5pt, draw=none] at (3) {\small $140$};
    %\node[below=5pt, draw=none] at (4) {\small $40$};
    %\node[below=5pt, draw=none] at (5) {\small $40$};
    %\node[above=5pt, draw=none] at (6) {\small $+\infty$};
    %\node[above=5pt, draw=none] at (7) {\small $+\infty$};
    %\node[below=5pt, draw=none] at (8) {\small $40$};
    %\node[below=5pt, draw=none] at (9) {\small $40$};
    %\node[above=5pt, draw=none] at (10) {\small $100$};
    %\node[below=5pt, draw=none] at (11) {\small $100$};

    % Arestas externas
    \draw[->][thick] (0) -- (1);
    \draw[->][thick] (2) -- (3);
    \draw[->][thick] (4) -- (5);
    \draw[->][thick] (6) -- (7);
    \draw[->][thick] (8) -- (9);
    \draw[->][thick] (10) -- (11);
    \draw[->][thick] (1) -- (2);
    \draw[->][thick] (1) -- (4);
    \draw[->][thick] (3) -- (6);
    \draw[->][thick] (3) -- (4);
    \draw[->][thick] (5) -- (2);
    \draw[->][thick] (5) -- (8);
    \draw[->][thick] (9) to[out=-60, in=-120] (4);
    \draw[->][thick] (9) -- (6);
    \draw[->][thick] (11) -- (6);
    \draw[->][thick] (9) -- (10);
    \draw[->][thick] (11) to[out=-60, in=-120] (8);
    \draw[->][thick] (7) to[out=60, in=120] (0);

    % Arestas internas (capacidades dos vértices)
    \draw[->, thick] (2) -- (3);
    \draw[->, thick] (4) -- (5);
    \draw[->, thick] (8) -- (9);
    \draw[->, thick] (10) -- (11);

    \end{tikzpicture}
    \caption{Grafo transformado}
    \label{fig:grafo-transformado}
\end{figure}

\subsection*{2.2 Modelo no LPSolver}
Para a implementação no LP Solver, o problema é formulado como um modelo de programação linear (PL), com o objetivo de maximizar o fluxo total de origem para destino, sujeito às restrições de capacidade e de conservação de fluxo.

\vspace{0.5em}

Neste caso, não é necessário dividir os vértices, uma vez que a linguagem do LP Solver permite introduzir explicitamente restrições internas em vértices, o que facilita a modelação direta com base na estrutura original do grafo.

\vspace{0.5em}

Como o foco do trabalho não recai sobre a justificação deste modelo, a formulação deste não será explicada em grande detalhe, mas inclui:

\subsubsection*{2.2.1 Variáveis de decisão e parâmetros}
\begin{itemize}
    \item $x_{ij}$: representam o fluxo que passa pelo arco $(i,j)$
\end{itemize}

\subsubsection*{2.2.2 Restrições}
\begin{itemize}
    \item Restrição de conservação do fluxo: Para todos os vértices intermédios \( v \in \{2, 3, 5, 6\} \), impõe-se a igualdade entre o total de fluxo que entra e o total de fluxo que sai:

\[
\sum_{j \in V} x_{jv} = \sum_{j \in V} x_{vj} \quad \forall v \in V \setminus \{O, D\}
\]

    De forma expandida, para os vértices do grafo:
        \begin{align*}
            &x_{O2} + x_{32} - x_{23} - x_{2D} = 0 && \text{(vértice 2)} \\
            &x_{O3} + x_{23} + x_{53} - x_{32} - x_{35} = 0 && \text{(vértice 3)} \\
            &x_{35} + x_{65} - x_{53} - x_{56} - x_{5D} = 0 && \text{(vértice 5)} \\
            &x_{56} - x_{6D} = 0 && \text{(vértice 6)} \\
            &x_{O2} + x_{O3} - x_{2D} - x_{5D} - x_{6D} = 0; && \text{(fluxo geral)}
        \end{align*}
    \item Restrição da capacidade dos vértices: 
\[
\sum_{i : (i,v) \in A} x_{iv} \leq c_v
\]em que $c_{v}$ representa a capacidade do vértice $v$.

    Para os vértices do grafo:
        \begin{align*}
            &x_{02} + x_{32} \leq 140 \\
            &x_{03} + x_{23} + x_{53} \leq 40 \\
            &x_{56} \leq 100
        \end{align*}
\end{itemize}

\subsubsection*{2.2.3 Função objetivo}
Maximizar o somatório dos fluxos que saem do vértice de origem ($O$), ou que chegam ao destino ($D$).
\begin{itemize}
    \item max: $\sum_{j : (j, D) \in A} \quad x_{jD}$, com $A = \{2,5,6\}$
\end{itemize}

\section*{3. Ficheiro de \textit{input}}
Como foram elaborados dois modelos, a testar em plataformas diferentes, foram naturalmente criados dois ficheiros de \textit{input} diferentes, os quais se apresentam a seguir.

\subsection*{3.1 Relax4}
\begin{figure}[H]
\begin{lstlisting}
12
18
1   2     0 1000000
3   4     0     140
5   6     0      40
7   8     0 1000000
9  10     0      40
11 12     0     100
2   3     0 1000000
2   5     0 1000000
4   7     0 1000000
4   5     0 1000000
6   3     0 1000000
6   9     0 1000000
10  5     0 1000000
10  7     0 1000000
12  7     0 1000000
10 11     0 1000000
12  9     0 1000000
8   1    -1 1000000
0
0
0
0
0
0
0
0
0
0
0
0
\end{lstlisting}
\end{figure}

\subsection*{3.2 LPSolver}
\begin{figure}[H]
\begin{lstlisting}
max : x5D + x2D + x6D;

xO2 + x32 - x23 - x2D = 0;
xO3 + x23 + x53 - x32 - x35 = 0;
x35 + x65 - x53 - x56 - x5D = 0;
x56 - x6D = 0;
xO2 + xO3 - x2D - x5D - x6D = 0;

xO2 >= 0; x2D >= 0;
xO3 >= 0; x42 >= 0;
x23 >= 0; x5D >= 0;
x32 >= 0; x56 >= 0;
x35 >= 0; x65 >= 0;
x53 >= 0; x6D >= 0;

xO2 + x32 <= 140;
xO3 + x23 + x53 <=40;
x56 <= 100;
\end{lstlisting}
\end{figure}

\section*{4. Ficheiro de \textit{output}}

\subsection*{4.1 Relax4}
\begin{figure}[H]
\begin{lstlisting}
 END OF READING
 NUMBER OF NODES = 12, NUMBER OF ARCS = 18
 CONSTRUCT LINKED LISTS FOR THE PROBLEM
 CALLING RELAX4 TO SOLVE THE PROBLEM
 ***********************************
 TOTAL SOLUTION TIME =  0. SECS.
 TIME IN INITIALIZATION =  0. SECS.
   1 2  180.
   3 4  140.
   5 6  40.
   7 8  180.
   9 10  40.
   2 3  140.
   2 5  40.
   4 7  140.
   6 9  40.
   10 7  40.
   8 1  180.
 OPTIMAL COST =  -180.
 NUMBER OF AUCTION/SHORTEST PATH ITERATIONS = 11
 NUMBER OF ITERATIONS =  8
 NUMBER OF MULTINODE ITERATIONS =  2
 NUMBER OF MULTINODE ASCENT STEPS =  1
 NUMBER OF REGULAR AUGMENTATIONS =  2
 ***********************************
\end{lstlisting}
\end{figure}

\subsection*{4.2 LPSolver}
\begin{figure}[H]
\begin{lstlisting}
Value of objective function: 180.00000000

Actual values of the variables:
x5D                             40
x2D                             140
x6D                             0
xO2                             140
x32                             0
x23                             0
xO3                             40
x53                             0
x35                             40
x65                             0
x56                             0
x42                             0
\end{lstlisting}
\end{figure}

\section*{5. Interpertação dos resultados obtidos}
O resultado obtido pelos dois modelos implementados (Relax4 e LP Solver) indica que o fluxo máximo possível entre o vértice de origem $O = 1$ e o vértice de destino $D = 4$ é de 180 unidades. Esta conclusão é suportada tanto pelo custo ótimo devolvido pelo Relax4 (com valor $-180$, indicando fluxo de 180 unidades), como pelo valor da função objetivo do LP Solver (180).

\vspace{0.5em}

Ao analisar os fluxos individuais pelas arestas (conforme apresentado no output, na secção 4.), podemos decompor o fluxo ótimo da seguinte forma:
\begin{itemize}
    \item Caminho 1 → 2 → 4: Neste caminho passam 140 unidades de fluxo, que correspondem exatamente à capacidade do vértice 2. Este fluxo segue diretamente da origem para o destino, com um único vértice intermédio.
    
    \item Caminho 1 → 3 → 5 → 4: Este caminho transporta as restantes 40 unidades de fluxo, explorando o outro ramo da rede. O vértice 3 está limitado a 40 unidades, que são todas utilizadas. O fluxo passa depois por 5 (também com capacidade de 40), e segue para o destino final.
\end{itemize}

Deste modo, o fluxo total de 180 unidades é obtido pela soma dos dois caminhos anteriores:
\[
140 \text{ (via 2)} + 40 \text{ (via 3 e 5)} = 180
\]

Este resultado confirma que:
\begin{itemize}
    \item Os vértices 2, 3 e 5 estão a operar no seu limite de capacidade (que segue de acordo com as restrições de capacidade dos vértices, estabelecidas anteriormente);
    \item O vértice 6 não é utilizado na solução ótima, uma vez que os caminhos que o incluem não são vantajosos sob as restrições das capacidades disponíveis;
    \item A rede original permite redundância de caminhos, o que é essencial para atingir o fluxo máximo respeitando todas as limitações.
\end{itemize}

Graficamente, se anotarmos o fluxo sobre o grafo $G$, temos:
\begin{figure}[H]
\centering
\begin{tikzpicture}[scale=1.1, every node/.style={draw, circle, minimum size=7mm, inner sep=0pt}]
    % Nós
    \node (O) at (0,1) {1};
    \node (A) at (2,2) {2};
    \node (B) at (2,0) {3};
    \node (D) at (4,2) {4};
    \node (C) at (4,0) {5};
    \node (F) at (6,1) {6};

    % Capacidades
    \node[above=4pt, draw=none] at (O) {\small $+\infty$};
    \node[above=4pt, draw=none] at (A) {\small $140$};
    \node[below=10pt, draw=none] at (B) {\small $40$};
    \node[above=4pt, draw=none] at (D) {\small $+\infty$};
    \node[below=10pt, draw=none] at (C) {\small $40$};
    \node[above=4pt, draw=none] at (F) {\small $100$};

    % Arestas com fluxo
    \draw[->, thick] (O) -- (A) node[midway, above left=-2pt] {\footnotesize \textbf{140}};
    \draw[->, thick] (A) -- (D) node[midway, above] {\footnotesize \textbf{140}};
    \draw[->, thick] (O) -- (B) node[midway, below left=-2pt] {\footnotesize \textbf{40}};
    \draw[->, thick] (B) -- (C) node[midway, below] {\footnotesize \textbf{40}};
    \draw[->, thick] (C) -- (D) node[midway, below right=-2pt] {\footnotesize \textbf{40}};

    % Arestas restantes com cor cinzenta clara e traço fino (sem fluxo)
    \draw[->, draw=gray!50] (A) -- (B);
    \draw[->, draw=gray!50] (D) -- (C);
    \draw[->, draw=gray!50] (D) -- (F);
    \draw[->, draw=gray!50] (C) -- (F);
\end{tikzpicture}
\caption{Grafo original com fluxo ótimo anotado}
\label{fig:grafo-original-fluxo}
\end{figure}


Esta representação gráfica e quantitativa confirma a consistência do modelo implementado e a coerência dos resultados fornecidos pelos dois métodos computacionais utilizados. O grafo original é respeitado na sua estrutura e restrições, e a solução obtida é admissível, exata e ótima para o problema proposto.

\section*{6. Identificação do corte mínimo}

O conjunto de vértices { 2, 3 } (na notação original, antes do \textit{node-splitting}) separa completamente a origem O do destino D e tem capacidade total

\[
    c(2)+c(3)=140+40=180
\]

\begin{itemize}
    \item Se cortar o vértice 2 (cap 140) toda a rota direta O→2→D fica bloqueada.
    \item Se cortar o vértice 3 (cap 40) toda a rota alternativa O→3→5→D fica igualmente bloqueada.
    \item Qualquer outro corte que isole O de D exige pelo menos esses dois vértices ou soma uma capacidade maior (ex.: incluir 5 ou 6 acrescentaria mais 40 ou 100).
\end{itemize}

No grafo transformado, o corte mínimo corresponde aos dois arcos internos:

\[
    \text{3 → 4   (cap 140)}
\]
\[
\text{5 → 6  (cap  40)}
\]

A soma 140 + 40 = 180 coincide com:

\begin{itemize}
    \item o valor do fluxo máximo devolvido pelo Relax4, e
    \item o custo ótimo -180 (valor absoluto) no modelo com o arco de custo –1.
\end{itemize}

Portanto, { 2 e 3 } é a identificação do corte mínimo da rede.

\section*{7. Validação do modelo}
Validar o modelo foi uma parte essencial do trabalho, para garantir que tudo estava bem estruturado e que os resultados obtidos faziam realmente sentido. Ao longo do processo, foram aplicadas várias formas de verificação que nos deram confiança na correção do que foi desenvolvido.

\vspace{0.5em}

Começámos por rever manualmente os valores das capacidades dos vértices, para ter a certeza de que estavam de acordo com o número de inscrição escolhido. Confirmámos que todos os cálculos estavam corretos e que os valores obtidos foram bem aplicados nas restrições do modelo.

\vspace{0.5em}

Depois, verificamos a transformação do grafo original para o formato necessário ao Relax4, seguindo a abordagem necessária para acomodar as características da plataforma. Tivemos o cuidado de rever cada ligação para garantir que a estrutura da rede original era mantida e que só os arcos internos é que impunham limites ao fluxo.

\vspace{0.5em}

Para garantir ainda mais robustez, tal como foi explicado, implementámos o modelo também no LP Solver, formulando o problema como um modelo de programação linear. Esta abordagem alternativa permitiu-nos comparar os dois resultados e confirmar que ambos os modelos — apesar de estarem escritos de formas diferentes — deram exatamente o mesmo valor de fluxo máximo: 180 unidades. Esta coincidência não só validou a lógica da modelação como também confirmou que os ficheiros de input estavam bem construídos.

\vspace{0.5em}

No final, ao representar o fluxo ótimo sobre o grafo original, pudemos verificar visualmente que o modelo respeita as capacidades dos vértices e conserva corretamente o fluxo. Com isto, ficámos confiantes de que a solução obtida é válida, eficiente e está de acordo com o problema proposto. Torna-se óbvio que a solução respeita todas as regras do modelo de fluxos em redes com restrições nos vértices, mantendo a coerência com a topologia da rede original e maximizando o fluxo de forma eficiente. Este processo iterativo de validação confere confiança à robustez da solução apresentada.
\end{document}